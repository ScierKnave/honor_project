\documentclass[11pt]{article}
\usepackage{amsmath}
\usepackage{amssymb}
\usepackage{amscd}
\usepackage{graphicx}
\graphicspath{ {./images/} }

\title{Neural network knowledge distillation in tensor networks}
\author{Dereck Piché}
\date{\today}
\begin{document}
\maketitle
\begin{abstract}
\end{abstract}

\section{Introduction}
In the last few decades, the use of artificial neural networks 
has gained a lot of popularity in several application areas. Because 
of their usefulness, we would like to be able to use them in systems 
that have fewer computational resources (nested systems, for example). 
Thus, we would like to find methods to reduce their temporal and 
spatial complexity while keeping their capabilities. This is where 
knowledge distillation comes in. Suppose we have a trained neural 
network that performs well on a certain task T. Then we can say 
that this neural network has knowledge about this task T. We would 
like to be able to transfer the knowledge of an already trained neural 
network into a system that takes less space. The technical term given 
to the knowledge transfer process is knowledge distillation. Despite 
the youth of this avenue of study, it remains broad in scope. The 
project will therefore focus more specifically on knowledge distillation 
of artificial neural networks in tensor networks. 

We now need to briefly explain what a tensor network is. 
Tensor networks are mathematical objects originating from the modeling 
of quantum phenomena (see Richard Penrose and Feynman). Here we see tensors 
as a generalization of vectors and matrices where we have an arbitrary 
number of indices. It is important to mention that tensor networks are 
intimately linked to a graphical notation which allows us to manipulate 
them much more easily than if we were limited to a purely analytical notation.
 From this notation, researchers have created different types (or patterns or categories) 
 of tensor networks that possess certain distinct properties. Some of these patterns 
 (such as the MPS https://tensornetwork.org/mps/) possess properties that are crucial
  to our task. In particular, several factorization and optimization methods for 
  temporal and spatial complexity are known and studied for MPS (Matrix Product 
  State / Tensor Train). Thus, we could use these optimizations after distillation 
  to reduce the computational costs of neural networks. The project will consist 
  of analyses and experiments that will aim to clarify/advance this topic (in the 
  form of a report). We have not determined how theoretical or practical the project will be.

\section{Learning Tensors Networks and Deep Neural Networks}
First, we must clearly distinguish the two approaches before 
trying to convert one to the other. The main difference between 
Deep Neural Networks is in the order of the transformation. In the 
forward pass of a Learning Tensor Network, the transformation must 
be applied directly to the inputs before being actualised as elements 
of the tensor network. The learning procedure can then be applied. It 
is done "apriori". In Neural Networks, this is done at multiple steps 
in the network. 

\section{Convolutional Neural Networks}
We could reproduce the filters of CNN's by applying a 
Kronecker product of the feature map $[ 1, \ \ x ]^T $ on a 
group of locally bound functions. In turn, we use this to create a new map.

\section{Direct weight optimization approach}
We propose to distill knowledge into a student by directly 
taking a NN parent model and refactoring it's linear properties 
by use of tensor network methods. The method is less akin to teaching, 
and more with factorisation. In the paper 
https://arxiv.org/pdf/2207.02851.pdf (page 3), it is proposed that
 each vector of the weight matrices would be decomposed into a MPS.
  We would rather decompose the weight matrice directly. It was also 
  proposed that the each vector space mapping of the layers of the neural 
  network would be factorised used MPS. This is an interesting idea that 
  while simple deserves some experimental exploration.

\section{Locality}
Convolutional Neural Networks bringed a new idea to the table, the idea of locality. 
Structured data comes with position-based information. 
As such, we shall try to improve the by using tensor product. 

\section{Layer-by-layer approach}
It has been common to each layer of a neural network as a certain abstracted 
representation of the previous information and their for generalisibility. 
Thus, we propose to change the cost, we use a tensor for multilinear mapping 
and train each mapping individualy according to the output of it's paired set of layers. 
\begin{equation*}
\begin{CD}
    x @>>>H_1 @>>>H_2 @>>>(\dots)@>g>> \hat{y}
\end{CD}
\end{equation*}
Each hidden layer is of the form;
\begin{equation*}
    a^l = \sigma^l \bigl( W^l a^{l-1} \bigr)
\end{equation*}
In tensor layer form, it will be defined as 
\begin{equation*}
    a^l = T^l \cdot \Phi \bigl( a^{l-1} \bigr)
\end{equation*}b
The main difference is that in the tensor approach, the "heavy" part is done by the non-linear transformation while a little work is done with the linear mapping. The opposite is true with neural networks. Here, $\Phi(X)$ (X begin the input vector) is a tensor product of several identical non-linear mappings of each element $x_i$. Thus, we have 
\begin{equation*}
    \Phi(X) = \phi(x_1)\phi(x_2)\dots\phi(x_n)
\end{equation*}
Were each $\phi : \mathbb{R} \rightarrow \mathbb{R}^d$, and each $d > 1$. Thus, our $\Phi(X) : \mathbb{R}^n \rightarrow \mathbb{R}^{ (d \times)^{n-1}d} $. In other words, our $\Phi$ returns a tensor of order $n$, where each indices run from $1$ to $d$.

\section*{Expressivité des compositions de fonctions multilinéaires}
Soit $f(x) R^d \mapsto R^{2^d}$ une fonction qui prend un 
vecteur de variables et qui renvoit un tenseur de dimensions quelconques
dont les éléments contiennent les bases de la fonction multilinéaire des
variables du vecteur $x$. 

The vector given by $f(x)$ will return a tensor containing
every combination of monomes of degree $z$, where $z$ is the smallest
repeating variable in the input vector.

Soit $m(f, \theta): \ R^{2^d} \mapsto (R^d \mapsto R)$ une fonction qui effectue une 
combinaison linéaire des éléments de $f$ and returns an \textit{instance} of
a multilinear function. Let this \textit{instance} depend on $\theta$. 

Let $S_v$ be a set of $n_v$ variables.
Let $v*$ be a vector where each variable in the set $S_v$ is repeated $n_v$ times. 

Since the vector given by $f(x)$ will return a tensor containing
every combination of monomes of degree $z$ and $z = n_v$, the elements
of the ouput of $f(x)$ will return the basis for the space 
of $n_v$-degree polynomial over the set of variables $S_v$. 


\end{document}
This is never printed
