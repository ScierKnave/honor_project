\documentclass[11pt]{article}
\usepackage{amsmath}
\usepackage{amscd}
\title{Neural network knowledge distillation in tensor networks}
\author{Dereck Piché}
\date{\today}
\begin{document}
\maketitle
\begin{abstract}
This is the paper's abstract \ldots
\end{abstract}

\section{Introduction}
This is time for all good men to come to the aid of their party!

\section{Layer-by-layer approach}
It has been common to each layer of a neural network as a certain abstracted representation of the previous information and their for generalisibility. Thus, we propose to change the cost, we use a tensor mapping and train each mapping individualy.
\begin{equation*}
\begin{CD}
    x @>>>H_1 @>>>H_2 @>>>(\dots)@>g>> \hat{y}
\end{CD}
\end{equation*}
Each hidden layer is of the form;
\begin{equation*}
    a^l = \sigma^l \bigl( W^l a^{l-1} \bigr)
\end{equation*}
In tensor layer form, it will be defined as 
\begin{equation*}
    a^l = T^l \cdot \Phi \bigl( a^{l-1} \bigr)
\end{equation*}b
The main difference is that in the tensor approach, the "heavy" part is done by the non-linear transformation while a little work is done with the linear mapping. The opposite is true with neural networks. Here, $\Phi(X)$ (X begin the input vector) is a tensor product of several identical non-linear mappings of each element $x_i$. Thus, we have 
\begin{equation*}
    \Phi(X) = \phi(x_1)\phi(x_2)\dots\phi(x_n)
\end{equation*}
Were each $\phi : \mathbb{R} \rightarrow \mathbb{R}^d$, and each $d > 1$. Thus, our $\Phi(X) : \mathbb{R}^n \rightarrow \mathbb{R}^{ (d \times)^{n-1}d} $. In other words, our $\Phi$ returns a tensor of order $n$, where each indices run from $1$ to $d$.


\section{Going further with local transformations}
In previous works, the non-linear mappings were independant and equal for each $(x_i)$. It would be interesting if these linear mappings were codependant. Here is an example of transformation to desambiguate what we mean
\begin{equation*}
    \phi_i(x_{i}, x_{i+1}) = \begin{bmatrix}
                                x_{i} \\
                                x_{i} \cdot x_{i+1}
                            \end{bmatrix}
\end{equation*}
It would be interesting to apply this transformation for a set of randomised pairs before the training.

\section{On transformations}
This is it!

\begin{gather*}
T_{1} \phi _{1}( x) +T_{2} \phi _{2}( x) +k_{1}\\
\equiv \\
T_{3} \phi _{1}( \ T_{5} \phi _{1}( x) +T_{6} \phi _{2}( x) +k_{3} \ ) +T_{4} \phi _{2}( \ T_{75} \phi _{1}( x) +T_{8} \phi _{2}( x) +k_{4} \ ) +k_{2}\\
\equiv T_{3} \phi _{1}( \ \lambda _{1}( x) \ ) +T_{4} \phi _{2}( \ \lambda _{2}( x) \ ) +k_{2}\\
\end{gather*}
Then we have with derivatives

\begin{gather*}
\frac{d}{dx}[ \ T_{1} \phi _{1}( x) +T_{2} \phi _{2}( x) +k_{1} \ ]\\
\equiv \\
\frac{d}{dx}[ \ T_{3} \phi _{1}( \ \lambda _{1}( x) \ ) +T_{4} \phi _{2}( \ \lambda _{2}( x) \ ) +k_{2} \ ]\\
\end{gather*}
Which is saying that

\begin{gather*}
T_{1}\frac{d}{dx} \ \phi _{1}( x) +T_{2}\frac{d}{dx} \phi _{2}( x)\\
\equiv \\
\frac{d}{dx} T_{3} \phi _{1}( \ \lambda _{1}( x) \ ) +\frac{d}{dx} T_{4} \phi _{2}( \ \lambda _{2}( x) \ )\\
\equiv T_{3}\frac{d\phi _{1}}{d\lambda _{1}}\frac{\lambda _{1}( x)}{dx} +\frac{d}{dx} T_{4}\frac{d\phi _{2}}{d\lambda _{2}}\frac{\lambda _{2}( x)}{dx}\\
\equiv T_{3}\frac{d\phi _{1}}{d\lambda _{1}}\frac{\lambda _{1}}{dx} +T_{4}\frac{d\phi _{2}}{d\lambda _{2}}\frac{\lambda _{2}}{dx}\\
\equiv T_{3}\frac{d\phi _{1}}{d\lambda _{1}}\left[ T_{5}\frac{d}{dx} \phi _{1}( x) +T_{6}\frac{d}{dx} \phi _{2}( x)\right] +T_{4}\frac{d\phi _{2}}{d\lambda _{2}}\left[ T_{7}\frac{d}{dx} \phi _{1}( x) +T_{8}\frac{d}{dx} \phi _{2}( x)\right]\\
\equiv T_{3} T_{5} T_{4} T_{7}\frac{d\phi _{1}}{d\lambda _{1}}\frac{d\phi _{2}}{d\lambda _{2}}\frac{d}{dx} \phi _{1}( x) +T_{3} T_{6} T_{4} T_{8}\frac{d\phi _{1}}{d\lambda _{1}}\frac{d\phi _{2}}{d\lambda _{2}}\frac{d}{dx} \phi _{2}( x)\\
\equiv \ T'_{1}\frac{d\phi _{1}}{d\lambda _{1}}\frac{d\phi _{2}}{d\lambda _{2}}\frac{d}{dx} \phi _{1}( x) +T'_{2}\frac{d\phi _{1}}{d\lambda _{1}}\frac{d\phi _{2}}{d\lambda _{2}}\frac{d}{dx} \phi _{2}( x)\\
\end{gather*}
Thus, we can see that it will only possible in this case if the set of do not form a constant!
In other words, they must not be be cancellable!. We can generalise this to higher order transformations! Their derivatives must not be cancellable! It is a requirement!


\bibliography{simple}

\end{document}
This is never printed
