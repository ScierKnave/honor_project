\documentclass[11pt]{article}
\usepackage{amsmath}
\usepackage{amssymb}
\usepackage{amscd}
\usepackage{graphicx}
\usepackage{euler}
\graphicspath{ {./images/} }

\title{Neural network knowledge distillation in tensor networks}
\author{Dereck Piché}
\date{\today}


\begin{document}
\maketitle
\begin{abstract}
\end{abstract}

\section{Introduction}
void


\section{Knowledge Distillation}
Knowledge Distillation is a machine learning practice which involves
taking a trained model and using it's parameters to train another one.
The already trained model is refered to as the "teacher", and 
the model in which his "knowledge" is to be distilled is called the "student".

\subsection{Response-Based Knowledge Distillation}
The first approach is to look exclusively at the outputs
of the student and teachers. Now, we apply the logits
of our functions element-wise to the outputs and the softmax.
\begin{equation}
    softmax(v_i) = \frac{e^{v_i}}{\sum_{j}{e^{v_i}}}
\end{equation}
The softmax function's objective is to transform the logits
into probability distributions for the different classes.
We will now apply a loss function $L$ to these two functions.
Since we are trying to reduce the divergence between 
two distributions, we will use the Kullback-Leibler divergence loss
\begin{equation}
    KL(P, Q) = \frac{1}{n} \sum_i^n Q \frac{log_e(Q)}{log_e(P)}
\end{equation}
Here, $Q$ is the distribution we are aiming for and $P$ is the one we have.



\section{Tensor Networks}
Tensor Networks come from the study of quantum phenomena. They 
started being used recently as machine learning models.

Consider a input vector $x$.

\subsection{Tensor Products and Transformations}

Consider an input vector $x \in R^d$. Let $\phi(x) : R \mapsto R^{d_{\phi}}$.  
Then, let us take the tensor product of apply $\phi(x)$ applied 
to every element of the vector $x$.
\begin{equation}
    \Phi(x) = \phi(x_1) \otimes  \phi(x_2) 
                \otimes (\dots) \phi(x_d)
\end{equation}
We obtain a tensor $\Phi(x)$, of which the sum of it's element 
contain the basis of a space of products of the elements of the transformed
elements of $\phi(x)$.

In order to reduce the abstractness of this statement, we can take say
that $\otimes$ refers to the Kronecker Product.

A particular dimension is of interest here, the transformation

\begin{equation}
    \phi(x) = 
    \begin{bmatrix}
        1 \\
        x
    \end{bmatrix}
\end{equation}
With, this particular transformation, the Kronecker Product gives us a 
$\Phi(x)$ which is a matrix where each element is a basis of the 
space of multilinear functions on the elements of the vector $x$.

\subsection{Linear Combinations}
How do we use $\Phi(x)$, how does it become useful? Well, it becomes useful 
when we take a linear map of it's elements. Let our $\Phi(x)$ be an 
arbitrely shaped tensor of $d_{\Phi}$ dimensions. The final form of our 
function $f(x)$ will be
\begin{equation}
    f(x) = \sum_{i}^{d_{\Phi}} \theta_i [\Phi(x)]_i
\end{equation}
Now, we have a function that can, under some choice of transformation
$\phi$, become extremely expressive.

\subsection{The \textit{Matrix Product State} Tensor Network}

\subsection{Expressivity of MPS combinations with transformation $[1,x]^t$}
Let $x \in R^d$.
Let $f(x) R^d \mapsto R^{z^d}$ function that returns a vector $v$.
Let $f(x)$ be defined such that $v$ contains every 
$x_i$-variate monome of degree $<= z$ where the variables are the elements of $x$. 

Let 
\begin{equation}
    \Theta = \{ \theta_1, \theta_2, (\dots), \theta_{d_{theta}} \}
\end{equation}
\begin{equation}
    M(f(x), \Theta)
    = 
    \begin{bmatrix}
        m(f(x), \theta_1) \\
        m(f(x), \theta_2) \\
        (\dots) \\
        m(f(x), \theta_{d_{theta}}) \\
    \end{bmatrix}
\end{equation}

Where $m(f(x), \theta): \ R^{2^d} \mapsto (R^d \mapsto R)$ is of the form
\begin{equation}
    m(f(x), \theta_i)
    = 
    m(v, \theta_i)
    =
    \sum_j \theta_{i,j} \cdot v_j
\end{equation}

We can choose $\Theta$ such that 
\begin{equation}
    M(f(x), \Theta)
    = 
    \begin{bmatrix}
        x_1 \\
        x_1 \\
        (\dots) \\
        x_d \\
    \end{bmatrix}
    = \lambda
\end{equation}

We can rewrite $\lambda$ as
\begin{equation}
    \lambda 
    = 
    \begin{bmatrix}
        \varepsilon_{1, 1} &
        (\dots) &
        \varepsilon_{1, z} &
        \varepsilon_{2, 1} &
        (\dots) &
        \varepsilon_{2, z} &
        (\dots) &
        \varepsilon_{d, z} \\
    \end{bmatrix}^T
\end{equation}

Every monome of $\zeta$  is of the form
\begin{equation}
    x_{1}^{k_1}x_{2}^{k_2}(\dots)x_{d}^{k_d}
\end{equation}
where $\sum_{i}k_i \leq z$. 
However, for every $K = {k_1, k_2, k_d}$ meeting this condition,
we can rewrite as 
\begin{equation}
    \left( \prod_{i_1=1}^{k_1}\varepsilon_{1, i_1} \right)
    \left( \prod_{i_2=1}^{k_2}\varepsilon_{2, i_2} \right)
    \left( \dots \right)
    \left( \prod_{i_d=1}^{k_d}\varepsilon_{d, i_d} \right)
\end{equation}
However, we clearly see that this term is a multilinear 
monome of the variables in $\lambda$.

\end{document}
Person 1: (while waiving figoriously)
    HELLO CAN YOU SEE ME!
    HEYYY! 

Person 2:
    For the last time, they can't see you... 
    YOU'RE OUTSIDE THE DOCUMENT DELIMITERS!

Person 1: 
    I feel so alone... 

Person 2: 
    Bro I'm here...